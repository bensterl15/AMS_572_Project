It has been well established by the United States Environmental Protection Agency (EPA) and other independent organizations that average temperature is increasing across the globe\cite{temp_anomaly}. This study examines monthly temperature data from two stations: Salt Lake City International Airport, Utah and Rye Patch Dam, Nevada. These stations were chosen because they are roughly at the same latitude ($40.790^{\circ}$N and $40.498^{\circ}$N), longitude ($111.980^{\circ}$W and $118.316^{\circ}$W), and elevation ($1287.8$m and $1260.3$m, for Salt Lake City and Rye Patch Dam, respectively). Our data is from the National Centers for Environmental Information's Global Summary of the Month (GSOM) dataset \cite{gsom_data}, which contains monthly data about major meteorological parameters at many locations across the country. Data for Rye Patch Dam begins in 1935, while data for Salt Lake City begins in 1948. Therefore, we only used years from 1948 to 2020 in our analysis. The dataset includes many parameters, but of particular interest to us were average temperature (${}^{\circ}$C), total minutes of sunshine, number of days with thunderstorms, and total precipitation (mm) per month.

The goal of our first hypothesis is to examine whether the effects of climate change statistically differ between these two locations; if they do, this would support claims that climate change is predominantly man-made\cite{obama_report}. We accomplished this by calculating the monthly temperature anomaly at each location and comparing the mean temperature anomalies of the two locations. The goal of our second hypothesis is to determine predictors of temperature in Salt Lake City. We use a multiple linear regression with temperature as the response variable and minutes of sunshine, days with thunderstorms, and precipitation as the predictor variables.