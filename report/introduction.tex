It has been well established by The United States Environmental Protection Agency and other independent organizations that average temperature is increasing across the country. This study examines monthly temperature data from two stations: Rye Patch Dam, Nevada, and Salt Lake City International Airport, Utah. These stations were chosen because they are roughly at the same latitude ($40.498^{\circ}$N and $40.790^{\circ}$N), longitude ($118.316^{\circ}$W and $111.980^{\circ}$W), and elevation ($1260.3$m and $1287.8$m, for Rye Patch Dam and Salt Lake City respectively). Our data is from the National Centers for Environmental Information \cite{gsom_data}, which contains monthly data about major meteorological parameters at many locations across the country. We chose Salt Lake City and Rye Patch Dam because of their similar geographical characteristics. Data for Rye Patch Dam begins in 1935, but data for Salt Lake City only reaches back to 1948, so we used only years from 1948 to 2020 in our analysis. The dataset includes many parameters, but of particular interest to us were average temperature, average precipitation, number of days with thunderstorms, and total minutes of sunshine.

The goal of our first hypothesis is to examine whether the effects of climate change statistically differ between these two locations; if they do, we may be able to infer that the climate change is predominantly man-made. We accomplished this by calculating the monthly temperature anomaly at each location and comparing the mean temperature anomalies of the two locations. The goal of our second hypothesis is to determine predictors of temperature. We use a multiple linear regression with temperature as the response variable and precipitation, days with thunderstorms, and minutes of sunshine as the dependent variables.